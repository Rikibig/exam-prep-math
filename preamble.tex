\usepackage{amsmath}
\usepackage{amssymb}
\usepackage{amsthm}
\usepackage{graphicx}
\usepackage{float}
\graphicspath{ {./images/} }
\usepackage{pdfpages}
\usepackage{hyperref}
\usepackage[danish]{babel}
\usepackage{tikz}
\usepackage{mathtools}

\usepackage{caption}

\usetikzlibrary {angles, quotes, through}
\usetikzlibrary{calc, tikzmark, intersections}

%\usepackage[e]{esvect}

\usepackage{tcolorbox}

\newcommand\gap[0]{\ensuremath{\hspace{1cm}}}

\newcommand\vektor[2]{\ensuremath{
    \begin{pmatrix}
        #1\\
        #2
    \end{pmatrix}
}
}

\newcommand\billede[3][1\textwidth]{\begin{figure}[H]
    \caption{#3}
    \includegraphics[width=#1]{#2}\label{#2}
    \centering
\end{figure}}

\newcommand\cel{\ensuremath{^\circ\mathrm{C}}}
\newcommand\grader{\ensuremath{^\circ}}
\newcommand\dx{\:\mathrm{d}x}
\newcommand\ud{\,\mathrm{d}}

\newcommand\lr{\ensuremath{\Leftrightarrow}}
\newcommand{\R}{\ensuremath{\mathbb{R}}}
\newcommand{\C}{\ensuremath{\mathbb{C}}}
\newcommand\dmht[1]{\frac{\partial}{\partial #1}}

\newcommand\inccounter[1]{\setcounter{#1}{\arabic{#1}+1}}
\newcommand\facit[1]{\underline{\underline{#1}}}

\tcbuselibrary{theorems}

\newcounter{opgave}
\newtcbtheorem[use counter=opgave]{opgave}{Opgave}{separator sign none}{}
\newtcbtheorem[number within=opgave, number format=\alph]{del}{Delopgave}%
{theorem style=plain,colframe=white,coltitle=black,fontupper=\itshape}{}
\newtcbtheorem{formel}{Formel}{theorem style=break,colframe=blue!70!white,coltitle=black!90!white,fonttitle=\itshape}{lm}

\newtcbtheorem[number within=section]{eksempel}{Eksempel}{theorem style=plain, colframe=black, colback=white, %
colbacktitle=blue!30!white,coltitle=black!90!white,fonttitle=\itshape}{ex}
\newtcbtheorem[number within=section]{definition}{Definition}{theorem style=plain, colframe=black, colback=white, boxrule=0.25mm, sharp corners=all,%
colbacktitle=blue!30!white,coltitle=black!90!white,fonttitle=\bfseries}{df}

\newtcbtheorem[number within=section, use counter from=definition]{theorem}{Sætning}{sharp corners=all, theorem style=plain, colframe=black, colback=white,%
colbacktitle=blue!30!white,coltitle=black!90!white,fonttitle=\bfseries, boxrule=0.25mm}{th}

\newtcbtheorem[number within=section, use counter from=definition]{lemma}{Lemma}{sharp corners=all, theorem style=plain, colframe=black, colback=white,%
colbacktitle=blue!30!white,coltitle=black!90!white,fonttitle=\bfseries, boxrule=0.25mm}{lm}
